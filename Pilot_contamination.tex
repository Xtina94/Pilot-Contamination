 \documentclass[11pt]{book}

% \usepackage{amssymb}
% \documentclass[12pt,a4paper]{article}
% % \usepackage[subpreambles=true]{standalone}
% \usepackage[a4paper]{geometry}
% \usepackage[utf8]{inputenc}
% \usepackage[T1]{fontenc}
% \usepackage[light]{CormorantGaramond}
% \usepackage{import}
%
% \linespread{1.25}

\usepackage{amsmath, amssymb, amsfonts, amsthm, mathtools}
\newcommand{\R}{\mathbb{R}}
\newcommand{\Z}{\mathbb{Z}}
\newcommand{\argmin}{\arg\!\min} % AlfC
\newcommand{\argmax}{\arg\!\max} % AlfC
\DeclareMathOperator{\tr}{tr}
\newcommand{\irchi}[2]{\raisebox{\depth}{$#1\chi$}}
\newcommand{\rchi}{{\mathpalette\irchi\relax}}
\usepackage{algorithm}
\usepackage{algpseudocode}
\usepackage{float}

\usepackage{microtype} %improves the spacing between words and letters

\usepackage{lipsum}
\usepackage{threeparttable}
\usepackage{tabularx}
\usepackage{multirow}
\usepackage{booktabs}
\newcommand{\tabitem}{~~\llap{\textbullet}~~}
\usepackage{graphicx}
\graphicspath{ {./pics/} {./eps/}}
\usepackage{epsfig}
\usepackage{epstopdf}

%%%%%%%%%%%%%%%%%%%%%%%%%%%%%%%%%%%%%%%%%%%%%%%%%%
%% COLOR DEFINITIONS
%%%%%%%%%%%%%%%%%%%%%%%%%%%%%%%%%%%%%%%%%%%%%%%%%%
\usepackage[dvipsnames]{xcolor} % Enabling mixing colors and color's call by 'svgnames'
%%%%%%%%%%%%%%%%%%%%%%%%%%%%%%%%%%%%%%%%%%%%%%%%%%
\definecolor{MyColor1}{rgb}{0.2,0.4,0.6} %mix personal color
\definecolor{MyColor2}{HTML}{A30073}
\definecolor{MyColor3}{HTML}{A3008F}
\newcommand{\textb}{\color{Black} \usefont{OT1}{lmss}{m}{n}}
\newcommand{\blue}{\color{MyColor1} \usefont{OT1}{lmss}{m}{n}}
\newcommand{\blueb}{\color{MyColor1} \usefont{OT1}{lmss}{b}{n}}
\newcommand{\red}{\color{LightCoral} \usefont{OT1}{lmss}{m}{n}}
\newcommand{\green}{\color{Turquoise} \usefont{OT1}{lmss}{m}{n}}
%%%%%%%%%%%%%%%%%%%%%%%%%%%%%%%%%%%%%%%%%%%%%%%%%%

%%%%%%%%%%%%%%%%%%%%%%%%%%%%%%%%%%%%%%%%%%%%%%%%%%
%% FONTS AND COLORS
%%%%%%%%%%%%%%%%%%%%%%%%%%%%%%%%%%%%%%%%%%%%%%%%%%
%    SECTIONS
%%%%%%%%%%%%%%%%%%%%%%%%%%%%%%%%%%%%%%%%%%%%%%%%%%
\usepackage{titlesec}
\usepackage{sectsty}
%%%%%%%%%%%%%%%%%%%%%%%%
%set section/subsections HEADINGS font and color
\newcommand*{\myfont}{\fontfamily{cmr}\selectfont}
\chapterfont{\myfont \color{MyColor1}}
\sectionfont{\myfont \color{MyColor1}}
\subsectionfont{\myfont\color{MyColor1}}  % sets colour of sections

%set section enumerator to arabic number (see footnotes markings alternatives)
% \renewcommand\thechapter{\arabic{chapter}.} %define chapters numbering
% \renewcommand\thesection{\arabic{section}.} %define sections numbering
% \renewcommand\thesubsection{\thesection\arabic{subsection}} %subsec.num.

%%%%%%%%%%%%%%%%%%%%%%%%%%%%%%%%%%%%%%%%%%%%%%%%%%
%		CAPTIONS
%%%%%%%%%%%%%%%%%%%%%%%%%%%%%%%%%%%%%%%%%%%%%%%%%%
\usepackage{caption}
\usepackage{subcaption}
%%%%%%%%%%%%%%%%%%%%%%%%
\graphicspath{{./figures/}} %Setting the graphicspath
% \graphicspath{{./figures/G&D/}} %Setting the graphicspath
% \captionsetup[figure]{labelfont={color=Turquoise}}

%%%%%%%%%%%%%%%%%%%%%%%%%%%%%%%%%%%%%%%%%%%%%%%%%%
%		!!!EQUATION (ARRAY) --> USING ALIGN INSTEAD
%%%%%%%%%%%%%%%%%%%%%%%%%%%%%%%%%%%%%%%%%%%%%%%%%%
%using amsmath package to redefine eq. numeration (1.1, 1.2, ...)
%%%%%%%%%%%%%%%%%%%%%%%%
\renewcommand{\theequation}{\thesection\arabic{equation}}

%set box background to grey in align environment
\usepackage{etoolbox}% http://ctan.org/pkg/etoolbox
\makeatletter
\patchcmd{\@Aboxed}{\boxed{#1#2}}{\colorbox{black!15}{$#1#2$}}{}{}%
\patchcmd{\@boxed}{\boxed{#1#2}}{\colorbox{black!15}{$#1#2$}}{}{}%
\makeatother
%%%%%%%%%%%%%%%%%%%%%%%%%%%%%%%%%%%%%%%%%%%%%%%%%%

\makeatletter
\let\reftagform@=\tagform@
\def\tagform@#1{\maketag@@@{(\ignorespaces\textcolor{red}{#1}\unskip\@@italiccorr)}}
\renewcommand{\eqref}[1]{\textup{\reftagform@{\ref{#1}}}}
\makeatother
\usepackage{hyperref}
\hypersetup{colorlinks = true, linkcolor  = black}

%% LISTS CONFIGURATION %%
\usepackage{enumitem}
\setlist[enumerate,1]{start=1}
\renewcommand{\labelenumii}{\theenumii}
\renewcommand{\theenumii}{\theenumi.\arabic{enumii}.}
\newcommand{\cri}[1]{\textcolor{MyColor2}{\textbf{(Cri says: #1)}}}

%%%%%%%%%%%%%%%%%%%%%%%%%%%%%%%%%%%%%%%%%%%%%%%%%%
%% abbreviations:
%%%%%%%%%%%%%%%%%%%%%%%%%%%%%%%%%%%%%%%%%%%%%%%%%%
\usepackage[acronym]{glossaries}
\newacronym{pca}{PCA}{Principal Component Analysis}
\newacronym{mimo}{MIMO}{Multiple-input Multiple-output}
\newacronym{bs}{BS}{Base Station}
\newacronym{ut}{UT}{User Terminal}
\newacronym{csi}{CSI}{Channel State Information}
\newacronym{iot}{IoT}{Internet if Things}
\newacronym{m2m}{M2M}{Machine to Machine}
\newacronym{3gpp}{3GPP}{3rd Generation Partnership Program}
\newacronym{fdd}{FDD}{Frequency Division Duplex}
\newacronym{tdd}{TDD}{Time Division Duplex}
\newacronym{ce}{CE}{Channel Estimation}
\newacronym{ul}{UL}{Uplink}
\newacronym{sinr}{SINR}{Signal-to-Interference plus Noise Ratio}
\newacronym{sir}{SIR}{Signal-to-Interference Ratio}
\newacronym{evd}{EVD}{Eigenvalue Decomposition}
\newacronym{map}{MAP}{Maximum a-posteriori}
\newacronym{cdma}{CDMA}{Code Division Multiple Access}
\newacronym{snr}{SNR}{Signal to Noise Ratio}
\newacronym{svd}{SVD}{Singular Value Decomposition}

\makeatletter
\newcommand{\chapterauthor}[1]{%
  {\parindent0pt\vspace*{-25pt}%
  \linespread{1.1} \scshape#1%
  \par\nobreak\vspace*{35pt}}
  \@afterheading%
}
\makeatother


\begin{document}
\chapter{The Pilot Contamination problem}
\chapterauthor{Author Name (Cristina Gava)\\
Student ID 1155449}

\section{Overview}
Compared to existing cellular network infrastructures, nowadays there is an increasing need for technologies providing higher capacity. This comes from an always bigger demand for higher data rates in wireless mobile communication systems such as \gls{iot}, \gls{m2m} communication and other electronic services.

The current 4G cellular networks, \gls{3gpp} above all, were designed with the intention to support a peak spectral efficiency of $15$ bps/Hz, a bandwidth of $100$ MHz and an ultra-low latency \cite{Elijah2016}. Nonetheless, the estimated future traffic far exceeds the resources of the current 4G and so the need for 5G cellular networks.

One of the novelties of the 5G protocol which is beeing designed and refined in the present communication scenario is the \gls{mimo} system, a technology that focuses on the idea of implementing multiple antennas terminals in one device - or \gls{bs} - in order to enhance the quality and reliabilty of communication. Without going into details, one of the options for this system is the multiuser \gls{mimo} system, where and array of antennas serves a group of autonomous terminals at the same time. These terminals may be single-antenna cheap devices and the multiplexing throughput gains are shared among the \gls{ut}s \cite{Marzetta2010}.

In this type of system, the \gls{csi} has a crucial role, since forward-link data transmission needs that the base station know the forward channel, as well as the reverse-link data transmission require it to know the reverse channel. This is the reason why such things as pilot signals exist, but with them some problems might arise due to the contamination of such signals. What we mean to takle in this chapter is exactly to have a detailed look at those kind of problems, referred to as pilot contamination, and at a couple of main approaches to solve them.

\section{The pilot contamination problem}
In several works multi-user \gls{mimo} operations with a big excess of base station antennas are considered: in them the channel is estimated exploiting the feedback or channel reciprocity schemes through multiplexing over frequency - \gls{fdd} - or over time - \gls{tdd}. In \gls{tdd} a time-slot, over which the channel can be thought as constant, is divided between reverse-link pilots and forward-link data transmission. The pilots assume reciprocity to provide the \gls{bs} with an estimate of the forward channel, which in turn generates a linear pre-coder for data transmission \cite{Marzetta2010}. In the \gls{fdd} scenario, the division is made over the frequency and the system requires not only the estimation, but also feedbacks for both forward and reverse direction between the \gls{bs} and the \gls{ut}.

For this reason \gls{tdd} is considered a more suitable approach the \gls{fdd} when it comes to acquiring \gls{csi} in wireless systems \cite{Elijah2016} and following this line, we will focus on this system.

In \gls{tdd} the time pilots require is proportional to the number of terminals served, while the number of base station antennas does not influence it. At the same time, though, the number of terminals that can be served is limited by the coherence time. One of the principal findings in this sense is that the addition of \gls{bs} antennas always brings benefits to the SNR situation.

To simplify the observed scenario, several works focus on multi-user \gls{mimo} operations with an infinite number of base station antennas in a multicellular environment. In general, in a massive \gls{mimo} system, when the \gls{ut}s transmit their pilot sequences to the \gls{bs} in order to perform the channel estimation in the \gls{ul} training phase, every \gls{bs} not only learns the channel related to the intended \gls{ut}, but also fractions of the channels connected to other \gls{ut}s that happen to have pilots which are related to the ones used by the intended \gls{ut}.

% In this frame, orthogonal pilots would need length of at least $K \times L$ symbols (with $K = $ overall number of \gls{ut}s in a cell and $L = $ total number of cells in a system) due to frequency reuse factor of one, so non-orthogonal pilots across neighboring cells are used. At the same time, the use of $K \times L$ pilots is not feasible beacuse of short channel coherence times due to \gls{ut}s mobility \cite{Elijah2016}. Because of it, the problem of pilot contamination arises and it has been considered one of the main impirments in massive \gls{mimo} systems.

\subsection{\gls{ul} training}
The use of pilot in the \gls{tdd} scheme is related to the Uplink segment training. Considering the worst-case scenarion in this means assuming that all \gls{ut}s transmit synchronous pilot sequences of length $\tau$ symbols at the beginning of every coherence interval. Every cell then transmits a $\tau \times K$ orthogonal matrix $\textbf{S}_j = (\textbf{s}_{j1},\dots,\textbf{s}_{jk})$ which satisfies $\textbf{S}_j^T\textbf{S}_j = \tau \textbf{I}$. The received signal matrix at the $l_{th}$ \gls{bs} is then:
\begin{equation}
\textbf{Y}_l = \sqrt{p_u}\sum_{j=1}^{L}\textbf{D}_{l,j}^{1/2}\textbf{H}_{l,j}\textbf{S}_j^T + \textbf{N}_l
\label{eq:recSignal}
\end{equation}
with:
\begin{itemize}
  \item $\textbf{N}_l = $ the $M \times \tau$ additive noise matrix whose elements are are i.i.d. zero mean, circularly-symmetric complex gaussian $\mathcal{CN}(0,1)$ random variables;\\
  \item $p_u = $ the average transmit power at each user on the uplink and is a measure of pilot signal-to-noise ratio;\\
  \item
  \begin{equation}
    \textbf{D}_{l,j} =
    \begin{bmatrix}
      \beta_{l,1,j} &        &              \\
                    & \ddots &              \\
                    &        & \beta_{l,K,j}\\
    \end{bmatrix}
  \end{equation}
  with $\beta_{l,k,j}$ being the large scale fading coefficient;\\
  \item
  \begin{equation}
    \textbf{H}_{l,j} =
    \begin{bmatrix}
      h_{l,1,j,1} & \dots  & h_{l,k,j,1}\\
      \vdots      & \ddots & \vdots\\
      h_{l,1,j,M} & \dots & h_{l,k,j,M}\\
    \end{bmatrix}
  \end{equation}
  with $h_{l,k,j,m}$ being the small scale fading factor whose variables are $\mathcal{CN}(0,1)$.\\
\end{itemize}
\section{The connection with 5G protocol}
\cri{Cita paper 5th e paper del prof}
\section{The main sources of contamination}
Pilot conatmination can be related to two main causes: non-orthogonal pilot schemes and hardware impairments.
While the first source is the most common and known one, the second source has been considered only recently and is still gaining consideration.

\subsection{Non orthogonal pilot schemes}
Normally, in a multi-cell system where the same frequency is shraed by L users, pilots are assumed mutually orthogonal and hence the intra-cell intereference is considered negligible. However, when frequency reuse comes into play, these signals are affected by this intereference, resulting in pilot contamination. In this case, the expression for the received signal is tha same as in \ref{eq:recSignal} \cite{Elijah2016}.

The conclusion of inter-cell interference was reached already by Marzetta in \cite{Marzetta2010}, where he precisely excluded the other possible sources of intra-cell interference and shadow fading. The author starts from the already considered propagation model where the single-antenna terminals are randomly distributed over the cell and separated by hundreds of wavelengths. Under these assumptions, the propagation vectors between the base station and the different terminals would be uncorrelated, since for a sufficiently high number of elements in a base station array the typical angular spacing between any two terminals would be greater than the angular Rayleigh resolution of the array, resulting in asymptotically orthogonal propagation vectors for different terminals. In fact, it can be shown that the inner product between two propagation vectors of any two terminals has a standard deviation of $\sqrt{M}$ (with $M$ being the number of antennas at the \gls{bs}) and it can be related to the critical assumtpion that, as the number of base station antennas grows, this inner product grows at a lower rate that the inner products of propagation vectors with themselves.

Under this assumption, the intra-cell intereference, the fast fading and the additive receiver noise effect disappear, laving the inter-cell interference as the only remaning hobstacle.

In this context the author analiyses the transmission scenario where he considers:
\begin{itemize}
  \item Hexagonal cells;
  \item OFDM modulation;
  \item Unlimited number of antennas per \gls{bs};
  \item Single antennas terminals;
  \item \gls{tdd}.
\end{itemize}
Here the maximum number of terminals for which the \gls{bs} can learn the channel is limited by the time it takes to acquire the \gls{csi} from the moving terminals, specifically: $K_{max} = \tau N_{smooth}$ with $\tau$ the number of OFDM symbols and $N_{smooth}$ the time over which the channel response is constant. The implication is that in general pilots from different cells are non-orthogonal, unless, for $K$ pilots in the $l_{th}$ cell, $K\cdot L \leq \tau N_{smooth}$ is true. The main conclusion of this apporach is that the frequency reuse among cells makes this relation false, and so justifies the inter-cell interference as the main source of pilot contamination \cite{Marzetta2010}.
\subsection{Hardware impairments}
Some works studied the impairments that some hardware components in radio frequency chain are prone to; these impairments can affect the accuracy of the \gls{ce} with related pilot contamination. These works approached this contamination source modeling some sort of non-ideal behavior of each component, but with more attention to the system overall response. In general, one main study \cite{Bjornson2014} shows how the hardware impairments from the \gls{bs} are negligible, while the main component of it comes from the \gls{ut}s, which limit the capacity in massive \gls{mimo} systems af M grows large. In this scenario, the \gls{ul} non-ideal system model that takes into consideration the distorsion noise for the received signal $\textbf{y} \in \mathbb{C}$ at the \gls{bs}, considering the deterministic pilot signal $d \in \mathbb{C}$, is represented by:
\begin{equation}
  \textbf{y} = \textbf{g}(d + {\eta}_t^{UT}) + \mathbf{\eta}_r^{BS} + \textbf{v}
\end{equation}
Where the stochastic processes $\eta_t^{UT} \in \mathbb{C}$ and $\mathbf{\eta}_r^{BS} \in \mathbb{C}^{M\times 1}$ describe the impairments of the transmitter and the receiver hardware at the \gls{ut} and the \gls{bs} respectively. While the ergodic process which is the additive noise $\textbf{v} = \textbf{v}_{noise} + \textbf{v}_{interf} \in \mathbb{C}^{M\times 1}$ consists of independent receiver noise $\textbf{v}_{noise} \sim \mathcal{CN}(\textbf{0},\sigma_{BS}^2\textbf{I})$ and potential interference from other simultaneous transmissions.

\subsection{Effects of pilot contamination}
The effects that can be counted can be analyzed first by the formulation of \gls{sir} expressions that can subsequently be translated into achievable rates; As much as \gls{sinr} can be considered, which has been demonstrated to saturate as M tends to infinity both in the uplink and in the downlink segment \cite{Elijah2016} \cite{Buzzi}. At the same time system performance degrades visibly because pilot contamination when the inter-cell interference factor $\beta$, composed by shadow fading and path loss, increases \cite{Elijah2016}.
\section{Schemes for pilot contamination mitigation}
There are several existing methods which intend to eliminate or mitigate the effects of pilot contamination, in particular for the case of \gls{tdd} systems. The two main categories these solutions have been divided in are, namely, the pilot-based estimation approach and the subspace estimation approach. Below we will give a brief explanation of what the former method consists of, while we will present one solution for the latter.

\subsection{The pilot-based approach}
In this approach,the main idea is that non-orthogonal pilots are used across the cells, while orthogonal ones within the cell. Through this assumption it was possible, for example, to transmit pilot signals from different cells by shifting the pilot location in frames so that users in separate cells transmit Without overlapping in time. Another point of view is using precoding methods requiring a limited collaboration among the \gls{bs}s \cite{Elijah2016}.
\subsection{The subspace-based approach}
The strength of this approach is that it requires a very limited number of pilot symbols for operation. In these, \gls{csi} is obtained through the application of \gls{evd} on the covariance matrix of the received samples. Moreover, since is not true a priori that the desired channels are always stronger than the interfering channels, \gls{map} criteria were added to it in some solutions.

The solution we briefly consider here was elaborated by M\"uller \textit{et al.} in \cite{Ralf} and consists of a nonlinear channel estimation based on a subspace projection. One of the most important statements of the paper is the confutation of the conclusion drawn by Marzetta \textit{et al.} in \cite{Marzetta2010}, where they stated that the array gain cannot be achieved for channel estimation, but only for data detection. Disproving this, \cite{Ralf} uses this gain for the nonlinear channel estimation algorithm, as well as shows how pilot contamination is a consequence of linear channel estimation and not a fundamental effect that cannot be removed.

\subsubsection{System model}
The owrk considers a wireless communication channel in the \gls{ul}, with the channel bandwith smaller than the coherence bandwidth and the channel being frequency-flat, block-fading and narrowband. Let the number of trasmit antennas be T and the number of receiving antennas R $>$ T. In this context, we can elaborate the \gls{cdma} system:
\begin{equation}
  \mathbf{Y} = \mathbf{HX} + \mathbf{Z},
\end{equation}
where:
\begin{itemize}
  \item $\mathbf{X} \in \mathbb{C}^{T\times C}$ the transmitted data;
  \item $\mathbf{Y} \in \mathbb{C}^{R\times C}$ the received signal;
  \item $\mathbf{Z} \in \mathbb{C}^{R\times C}$ the additive noise;
  \item $\mathbf{H} \in \mathbb{C}^{R\times T}$ the channel matrix, whose columns denote the spreading sequences;
\end{itemize}
and we can deploy the well-know fact that there is no need to know the spreading sequences to demodulate the \gls{cdma}, to come up with a so-called blind approach described below.
\subsubsection{The proposed method}
The proposed algorithm states that it might be a good idea not to estimate the channel matrix $\mathbf{H}$, but directly consider the subspace channel:
\begin{equation}
  \mathbf{\tilde{Y}} = \mathbf{\tilde{H}X} + \mathbf{\tilde{Z}}
\end{equation}
and estimate the much smaller subspace channel matrix $\mathbf{\tilde{H}} \in \mathbb{C}^{t\times T}$.

This because only the largest eigenvalues of $\mathbf{Y}\mathbf{Y}^T$ are necessary to our purpose. In fact, considering the case of $T = 1$ active transmit antenna and looking for the matched filter $m^T$ such that the \gls{snr} at its output is maximum, we are indeed looking for maximizing the total received power normalized to the power gain of the filter. Mathematically:
\begin{equation}
  \mathbf{m} = \argmax_{\mathbf{m}}\frac{\mathbf{m}^T\mathbf{Y}\mathbf{Y}^T\mathbf{m}}{\mathbf{m}^T\mathbf{m}}
\end{equation}
And in such case, the maximum value is exactly given by the maximum eigenvalue of $\mathbf{Y}\mathbf{Y}^T$, for a result from the algebra.

If we now extend this idea to the case of multiple transmit antennas, we can see that the subspace of eigenvalues we need is given by the projection of the received signal onto the signal space basis $\mathbf{S}^T \in \mathbb{C}^{T\times R}$ which stands out by the matrix of left singular vectors obtained by \gls{svd} of $\mathbf{Y}$. To express it in a more formal way we have:
\begin{equation}
  \mathbf{Y} = \mathbf{U} \mathbf{\Sigma} \mathbf{V}^T
\end{equation}
with unitary matrices $\mathbf{U} \in \mathbb{C}^{R\times R}$ and $\mathbf{V} \in \mathbb{C}^{C\times C}$, and the diagonal matrix $\mathbf{\Sigma} \in \mathbb{C}^{R\times C}$ \cri{è corretto l'insieme di appartenenza di questi elementi?} with diagonal entries sorted in non-increasing order. Based on this the $\mathbf{U}$ matrix has the structure:
\begin{equation}
  \mathbf{U} = [\mathbf{S|\mathbf{N}}]
\end{equation}
with the null space basis being $\mathbf{N} \in \mathbb{C}^{R\times (R-T)}$.

The projection in question is thus expressed as:
\begin{equation}
  \mathbf{\tilde{Y}} = \mathbf{S}^T\mathbf{Y}
\end{equation}
In this scenario, a remark on the white noise can be done and precisely that if then we consider the case of massive \gls{mimo} scenario, where $R >> T$, then we see that the influence of it onto the signal subspace becomes negligible as $R \rightarrow \inf$. This because white noise is evenly distributed in all dimensions of the full space, which in turn is much bigger than the T-dimensional signal subspace.

What is left then is the co-channel interference from L neighboring cells, which typically is not white and jusitfies the relation for which the more colored the smaller the \textit{load} factor:
\begin{equation}
  \alpha = \frac{T}{R}
\end{equation}
Were we able to identify which singular values correspond to channel vectors from inside the cell, as opposed to the ones from neighboring cells, we could try to remove them by subspace projection. Moreover, we keep also in mind that any R-dimensional channel vector from any transmitter to the receive array in the interested cell is orthogonal to any other channel vector. Thus we can conclude that, in a cellular system with power-controlled handoff strategy, the norm of channel vectors from neighboring cells can never be bigger than the norm of interested-cell channel vectors; this means that we can identify the singular values belonging to transmitters internal to the cell by ordering them by magnitude \cite{Ralf}.
\subsubsection{Performance ananlysis}
In this section we will give an approximate estimation of the number of transmit antennas which is suitable for for the previous approach. \cri{rivedi questa cosa se poi non riesci a tirare fuori un valore per il numero di antenne}

In the frame of classical massive \gls{mimo} settings, with L finite, R,T $\rightarrow \inf$ and $0 \neq \alpha \ll 1$, the impairment process $\mathbf{Z}$ is composed by a white noise component $\mathbf{W}$ (having iid elements with zero mean and variance W) and the interference from L neighboring cells:
\begin{equation}
  \mathbf{Z} = \mathbf{W} + \mathbf{H}_I\mathbf{X}_I
\end{equation}
In our helps comes also the definition of the normalized coherence time:
\begin{equation}
  \kappa = \frac{C}{R}
\end{equation}
In the large antenna limit $R \rightarrow \inf$, if we consider that the singular values of the normalized noise $\frac{\mathbf{W}}{\sqrt{CW}}/$ are bounded by the interval
\begin{equation}
  \frac{1}{\sqrt{\kappa}} - 1 < x < \frac{1}{\sqrt{\kappa}} + 1
\end{equation}
we can say that, at most, the power of white noise being present in $\mathbf{\tilde{Y}}$ is:
\begin{equation}
  Noise_w = \left(\Big(\frac{\mathbf{W}}{\sqrt{CW}}\Big)\sqrt{CW}\right)^2 = CW\left(1+\frac{1}{\sqrt{\kappa}}\right)^2
\end{equation}
for one singular value, while it is
\begin{equation}
  TNoise_w
\end{equation}
in the worst case, where the T largest singular values of the noise affect the signal of interest.

To then calculate the \gls{snr} in $\mathbf{\tilde{Y}}$ we can give an espression for the total power of the received signal as $TRCP$, since the data signal $\mathbf{X}$ is iid zero mean with variance (which account for its power) $P$. The \gls{snr} can be thus lower bounded by:
\begin{equation}
  SNR \geq \frac{P}{W}\frac{R}{\left(1+ \frac{1}{\sqrt{\kappa}}\right)^2}
\end{equation}


As mentioned at the beginning of this section, there is also the co-channel interference from neighboring cells, which is not white but at the same time higly concentrated in certain subspaces and can be suppressed making use of the phase transition of spectra of large random matrices. Specifically, in \cite{Ralf2012} the authors show how, in a multiuser \gls{mimo} scenario, the eigenvalue distribution of $\mathbf{H}$, when properly scaled or shifted, separates into two separate non-overlapping bulks with each bulk being shaped very similar to the cases of pure scattering and pure line-of-sight. This comes to our help when trying to separate the co-channel interference (the equivalent of the pure scattering) from the normalized signal of interest $\frac{\mathbf{HX}}{\sqrt{TR}}$ (the equivalent of pure line-of-sight).

For this we start observing that in the appendix A of \cite{Ralf}, the signal of interest is shown to converge to:
\begin{equation}
  \mathcal{P} = \left[\frac{\kappa P}{\alpha}-2P\sqrt{\frac{\kappa^2 + \kappa}{\alpha}};\frac{\kappa P}{\alpha} + 2P\sqrt{\frac{\kappa^2 + \kappa}{\alpha}}\right]
\end{equation}
as $R \rightarrow \inf$ and for $\alpha \ll 1$.

We then proceed supposing the entries of the matrix of interfering signals $\mathbf{X}_I$ be iid with zero mean and variance $P$, while the entries of the matrix of interfering channels $\mathbf{H}_I$ be iid with zero mean and variance $I/P$. In this way the ratio accounts for the relative attenuation between the intercell users and users external to the cell and thus the empirical distribution of the squared singular values of the normalized co-channel interference ($\frac{\mathbf{H}_I\mathbf{X}_I}{\sqrt{TR}}$) converges to a limit distribution as well. This distribution, for $\alpha \ll 1$ is supported in the interval
\begin{equation}
  \mathcal{I} =  \left[\frac{\kappa I}{\alpha}-2I\sqrt{L\frac{\kappa^2 + \kappa}{\alpha}};\frac{\kappa I}{\alpha} + 2I\sqrt{L\frac{\kappa^2 + \kappa}{\alpha}}\right]
\end{equation}
Therefore, having obtained two intervals (and so relative upper bounds) on the signal received and the co-channel interference, we can make us of the statement in \cite{Ralf2012} to affirm that if these two intervals do not overlap, i.e.
\begin{equation}
  \mathcal{P}\cap \mathcal{I} = 0
\end{equation}
or better
\begin{equation}
  \frac{P}{I} > \frac{1+2\sqrt{\alphaL(1+\frac{1}{\kappa})}}{1-2\sqrt{\alpha(1+\frac{1}{\kappa})}} \approx{\alpha \ll 1} 1+2(1+\sqrt{L})\sqrt{\alpha\left(1+\frac{1}{\kappa}\right)}
\end{equation}
\section{Conclusions}
\bibliographystyle{ieeetr}
\bibliography{myBib}

\end{document}
